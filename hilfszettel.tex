\documentclass[10pt,landscape,a4paper]{article}
\usepackage[utf8]{inputenc}
\usepackage[ngerman]{babel}
\usepackage[T1]{fontenc}
%\usepackage[LY1,T1]{fontenc}
%\usepackage{frutigernext}
%\usepackage[lf,minionint]{MinionPro}
\usepackage{tikz}
\usetikzlibrary{shapes,positioning,arrows,fit,calc,graphs,graphs.standard}
\usepackage[nosf]{kpfonts}
\usepackage[t1]{sourcesanspro}
\usepackage{multicol}
\usepackage{wrapfig}
\usepackage[top=0mm,bottom=1mm,left=0mm,right=1mm]{geometry}
\usepackage[framemethod=tikz]{mdframed}
\usepackage{microtype}
\usepackage{pdfpages}
\usepackage{flowfram}


\let\bar\overline

\global\mdfdefinestyle{headerstyle}{%
	linecolor=gray,linewidth=1pt,%
	leftmargin=0mm,rightmargin=0mm,skipbelow=0mm,skipabove=0mm,
}

\newsavebox{\header}
\newlength{\headerheight}

\sbox{\header}{
    \begin{minipage}[t]{0.19\textwidth}
	\begin{mdframed}[style=headerstyle]
		\footnotesize
		\sffamily
		Hilfszettel zur Klausur\\
		von~Tim~S.,~Seite~\thepage~von~2
	\end{mdframed}
	\end{minipage}
}
\setlength{\headerheight}{\totalheightof{\usebox{\header}}}

\newflowframe{0.19\textwidth}{\textheight - \headerheight}{0pt}{0pt}[column1]
\newflowframe{0.19\textwidth}{\textheight}{0.2025\textwidth}{0pt}[column2]
\newflowframe{0.19\textwidth}{\textheight}{0.405\textwidth}{0pt}[column3]
\newflowframe{0.19\textwidth}{\textheight}{0.6075\textwidth}{0pt}[column4]
\newflowframe{0.19\textwidth}{\textheight}{0.81\textwidth}{0pt}[column5]


\begin{document}
\definecolor{myblue}{cmyk}{1,.72,0,.38}

\def\firstcircle{(0,0) circle (1.5cm)}
\def\secondcircle{(0:2cm) circle (1.5cm)}

\colorlet{circle edge}{myblue}
\colorlet{circle area}{myblue!5}

\tikzset{filled/.style={fill=circle area, draw=circle edge, thick},
	outline/.style={draw=circle edge, thick}}

\pgfdeclarelayer{background}
\pgfsetlayers{background,main}

\everymath\expandafter{\the\everymath \color{myblue}}
\everydisplay\expandafter{\the\everydisplay \color{myblue}}

\renewcommand{\baselinestretch}{.8}
\pagestyle{empty}

\makeatletter % Author: https://tex.stackexchange.com/questions/218587/how-to-set-one-header-for-each-page-using-multicols
\renewcommand{\section}{\@startsection{section}{1}{0mm}%
	{.2ex}%
	{.2ex}%x
	{\color{myblue}\sffamily\small\bfseries}}
\renewcommand{\subsection}{\@startsection{subsection}{1}{0mm}%
	{.2ex}%
	{.2ex}%x
	{\sffamily\bfseries}}
\makeatother
\setlength{\parindent}{0pt}

\newdynamicframe{0.3\textwidth}{\headerheight}{0pt}{\textheight - \headerheight}[authorbox]
\begin{dynamiccontents*}{authorbox}
	\begin{minipage}[t]{0.18\textwidth}
		\begin{mdframed}[style=headerstyle]
			\footnotesize
			\sffamily
			Hilfszettel zur Klausur\\
			von~Tim~S.,~Seite~\thepage~von~2
		\end{mdframed}
	\end{minipage}
\end{dynamiccontents*}
%\footnotesize
\small
\input{inhalt/aussagen}
\input{inhalt/mengen}
\input{inhalt/relationen}
\input{inhalt/abbildungen}
%\input{inhalt/beweise}
\input{inhalt/graphen}
%\input{inhalt/graphalgo}
%\input{inhalt/bool}
%\input{inhalt/formeln}
\end{document}
