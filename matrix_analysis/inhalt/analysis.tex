\section{方阵的级数与函数}
\subsection*{方阵级数}
$\textstyle\sum\limits_{k=0}^{\infty}A^k$收敛等价于$\lim\limits_{k\to\infty}A^k=0$,此时
$\textstyle\sum\limits_{k=0}^{\infty}A^k=(1-A)^{-1}$。
设$f(z)=\textstyle\sum\limits_{k=0}^{\infty}a_kz^k$收敛半径为$R$,$\rho(A)<R$时$f(A)$有意义。
\subsection*{$f(A)$的计算方法}
理论上:将Jordan块拆为$\lambda I+N$,使用二项式定理展开计算一个$Jordan$块的级数,因为$N$幂零项数很少。
对$A$进行Jordan分解,每个Jordan块都能求出级数,故可得到$f(A)$。\\
实际中使用零化多项式$m(x)$插值计算:令$f(x)=q(x)m(x)+r(x)$,余项次数低于$m(x)$,故可待定系数将其设出;
此后代入$m(x)$零点可解出$r(x)$,于是将$A$代入即得。若有重根应求导造出足量方程。
\subsection*{常见级数}
\begin{enumerate}
	\item $ e^A = \sum_{k=0}^{\infty} \frac{A^k}{k!}, R = \infty $
	\item $ \cos A = \sum_{k=0}^{\infty} (-1)^k \frac{A^{2k}}{(2k)!}, R = \infty $
	\item $ \sin A = \sum_{k=0}^{\infty} (-1)^k \frac{A^{2k+1}}{(2k+1)!}, R = \infty $
	\item $ \ln(I + A) = \sum_{k=1}^{\infty} (-1)^{k+1} \frac{A^k}{k}, R = 1 $
	\item $ (1 + A)^\alpha = \sum_{k=0}^{\infty} \binom{\alpha}{k} A^k, R = 1 $
\end{enumerate}
\subsection*{$e^A$的性质}
\begin{enumerate}
	\item $\det(e^A)=e^{trA}$,对A做Jordan可证。
	\item $e^{iA}=\cos(A)+i\sin(A)$,若$A$是实矩阵可通过计算实部和虚部计算$\cos(A)$和$\sin(A)$。
	\item $A^T=-A\Rightarrow (e^A)^{-1}=(e^A)^T$(第一类正交阵),因为$e^{A+A^T}=I$。
	\item $A^H=-A\Rightarrow (e^A)^{-1}=(e^A)^H$(酉)。
\end{enumerate}
