\section{矩阵分解}
\subsection*{Schur}
方阵$A=UTU^{H}$,当$A^{H}A=AA^{H}$(正规)时有$A=U\Lambda U^{H}$。
标准型对角线元素为$A$全部特征值。证明基本思路为取单位特征向量构造酉阵,此后计算并归纳。
正规矩阵的Schur求法:求出所有特征向量并标准正交化得到$U$。
\subsection*{谱分解}
竖切正规矩阵Schur分解的$U$阵得到$A=(\xi_i)\Lambda(\overline{\xi_i})^T=\sum\xi_i\lambda_i\overline{\xi_i}$。
\subsection*{Jordan}
方阵$A=PJP^{-1}$,算法为:
\begin{enumerate}
	\item 求$A$的特征多项式,得到特征值
	\item 求$A$的最小多项式$\prod (\lambda_i - A)^{\beta_i}$,$\beta_i$为$\lambda_i$对应最大Jordan块大小
	\item 对于每个特征值,求$rank((\lambda_i I - A)^k)$,$k=1\dots\beta_i$。计算$n_k=nullity(\lambda_i I - A)^k$,
	      $n_1$为属于$\lambda_i$的Jordan块总数,$n_{k+1}-n_{k}$为尺寸不小于$k$的Jordan块个数。
	      原因是$(\lambda I - J)^i$与$(\lambda I - A)^i$相似,秩相等,而对于$(\lambda I - J)$每自乘一次,当前幂次阶以上的Jordan块就会多一个全零行。
	\item 求变换矩阵$P$。变形为$AP=PJ$后可列出$P$各列向量满足的方程。
\end{enumerate}
\subsection*{SVD}
$A^{m\times n}=U_{m}\left( \begin{smallmatrix}\Sigma^{r\times r}&O\\O&O\end{smallmatrix}\right)V_n^H,\Sigma=\left(\begin{smallmatrix}\sigma_1&&\\ & \ddots & \\ && \sigma_r\end{smallmatrix}\right) $,
奇异值$\sigma_i=\sqrt{\lambda_i}\neq0$,$\lambda_i$是$A^HA$的非零特征值,编号根据值降序排列。
证明时先构造$V$,由于$A^HA$正规,Schur得$V^HA^HAV=diag(\Sigma^2,O_{n-r})$。
竖切$V=(V_1^{n\times r},V_2^{n\times (n-r)})$,代入Schur计算有$V_1,V_2$与$A,\Sigma$的关系。
取$U_1^{m\times r}=AV_1\Sigma^{-1}$,满足$U^{H}_1U_1=I_r$,因此$U_1$各列向量标准正交。
将$U_1$增补为$U^{m\times m}$即为所需的$U$,可代入SVD验证。
\subsection*{极分解}
方阵$A = P\tilde{U} = U\tilde{Q}$,
其中$P,Q$Hermite半正定。从SVD开始配凑中间的$I$为酉阵与其共轭转置乘积得到。
一阶时为$x=re^{i\theta}$。
\subsection*{满秩分解}
注意到$A$中线性相关的情况后直接列写。
如$A=(\alpha_1,\alpha_2,\alpha_1+\alpha_2)=(\alpha_1,\alpha_2)\begin{psmallmatrix}
		1&0&1\\0&1&1
	\end{psmallmatrix} $
\subsection*{LR分解}
$A=P\left(\begin{smallmatrix}
			I_r&O\\O&O
		\end{smallmatrix}
	\right)Q=P\left(\begin{smallmatrix} I_r\\ O \end{smallmatrix} \right)\left(\begin{smallmatrix} I & O \end{smallmatrix} \right)
	Q=LR$

\section{矩阵分解相关}
\subsection*{Sylvester降幂公式}
对于$A\in \mathbb{C}^{m\times n},B\in \mathbb{C}^{n\times m},m\geq n$,有
$\det(\lambda I_m-AB)=\lambda^{m-n}\det(\lambda I_n-BA)$。
结合满秩分解可以用于降阶计算高阶特征多项式。
证明需要注意到\[
	\begin{aligned}
		\left(
		\begin{smallmatrix}
			I & B \\ O & I
		\end{smallmatrix}
		\right)\left(
		\begin{smallmatrix}
			O & O \\ A & AB
		\end{smallmatrix}\right)
		 & =\left(
		\begin{smallmatrix}
			BA & BAB \\ A & AB
		\end{smallmatrix}\right) \\
		 & =\left(
		\begin{smallmatrix}
			BA & O \\ A & O
		\end{smallmatrix}\right)\left(
		\begin{smallmatrix}
			I & B \\ O & I
		\end{smallmatrix}\right)
	\end{aligned}
\]再计算由此导出相似矩阵的特征多项式。
\subsection*{Carley-Hamilton定理}
对于$n$阶方阵A,其特征多项式$p(\lambda)=\det(\lambda I - A)$是它的一个零化多项式。
使用Schur或Jordan将$A$化为上三角矩阵后计算可证。
\subsection*{最小多项式}
次数最低的首一零化多项式。是所有零化多项式的因子,这一点可使用多项式整除求余证明。
\subsection*{广义特征向量}
$(A-\lambda I)^k\xi=0,(A-\lambda I)^{k-1}\xi\neq 0$
则称$\xi$为$A$深度为$k$的广义特征向量。
令$\xi_i=(A-\lambda I)^{k-i}$,那么有
$(A-\lambda I)^{n\times n}(\xi_1\dots x_k)^{n\times k}=(\xi_1\dots x_k)^{n\times k}J_{\lambda=0}^{k\times k}$
\subsection*{相似对角化}
方阵可相似对角化等价于其特征向量线性无关,因为$P^{-1}AP=\Lambda,AP=P\Lambda$,
根据特征值定义可证明;亦等价于最小多项式无重根,因为最大Jordan块大小为1。
\subsection*{典型正规矩阵}
$U,A^H=A,A^H=-A,\Lambda$
\subsection*{奇异向量}
SVD变形有$AV=U\left(\begin{smallmatrix}\Sigma & O \\ O & O\end{smallmatrix}\right)$,
竖切$U=(\eta_1\dots\eta_m),V=(\xi_1\dots\xi_n)$后展开计算,
% $A(\xi_1\dots\xi_n)=(\eta_1\dots\eta_m)\left(\begin{smallmatrix}\Sigma & O \\ O & O\end{smallmatrix}\right)$。
于是$A\xi_i=\eta_i\sigma_i,1\leq i\leq r$,称$\xi_i$为右奇异向量,$\eta_i$为左奇异向量。
此外$A\xi_i=O,r\leq i\leq n$。
\subsection*{基于SVD的矩阵压缩}
对SVD分解式竖切$U$,横切$V^H$有
$A=(\eta_1\dots\eta_m)\left(\begin{smallmatrix}\Sigma & O \\ O & O\end{smallmatrix}\right)(\xi_1^H\dots\xi_n^H)^T$,
展开得$A=\textstyle\sum\limits_{i=1}^{r}\eta_i\sigma_i\xi_i^H$,故大奇异值项占主要成分。
