\section{空间与变换}
\subsection*{线性空间}
对$V,+,\cdot$,在
\begin{enumerate}
	\item $(\alpha+\beta)+\gamma = \alpha + (\beta+\gamma)$
	\item $\alpha+\beta=\beta+\gamma$
	\item $\exists\theta\forall\alpha,\alpha+\theta=\alpha $
	\item $\forall\alpha\exists\gamma,\alpha+\gamma=\theta $
	\item $(\alpha\cdot k_1 )\cdot k_2 = \alpha\cdot(k_1k_2)$
	\item $\alpha\cdot(k_1+k_2)=\alpha\cdot k_1+\alpha\cdot k_2$
	\item $(\alpha_1+\alpha_2)\cdot k =\alpha_1\cdot k + \alpha_2\cdot k$
	\item $\alpha \cdot 1 = \alpha$
\end{enumerate}
中满足前四条的$\langle V,+ \rangle$为交换群,满足全部的$\langle V,+,\cdot \rangle$为线性空间。
\subsection*{维数、基底与坐标}
$\forall \beta\in V, \exists ! k_1\dots k_n\in F ,\beta = \textstyle\sum\limits_{i=1}^{n}\alpha_i k_i$
表示$n$维空间$V$下$\beta$在基底$\left(\begin{smallmatrix}
			\alpha_1\dots\alpha_n\end{smallmatrix}\right) $下的坐标为
$\left(\begin{smallmatrix}
			k_1\dots k_n\end{smallmatrix}\right)^T$。
如$V=\mathbb{R}^{2\times 3},F=\mathbb{R}$的一组基底是将$2\times 3$零矩阵逐个元素代入1得到的六个矩阵,
按照$\beta = \left(\begin{smallmatrix}
			\alpha_1\dots\alpha_n\end{smallmatrix}\right)
	\left(\begin{smallmatrix}
			k_1\dots k_n\end{smallmatrix}\right)^T$计算时可先将矩阵拉直为列向量。
\subsection*{换基底}
$\left(\begin{smallmatrix}
			\alpha_1\dots\alpha_n\end{smallmatrix}\right)P=\left(\begin{smallmatrix}
			\beta_1\dots\beta_n\end{smallmatrix}\right)$中$P$为基$(\alpha_{i})$到$(\beta_i)$的过渡矩阵。
于是$\beta=(\alpha_i)X=(\beta_i)Y=(\alpha_i)PX$导出$X=PY$。
\subsection*{内积空间}
对线性空间$\langle V,+,\cdot \rangle$和运算$\langle,\rangle$满足
\begin{enumerate}
	\item $\langle\beta,\alpha\rangle = \overline{\langle\alpha,\beta\rangle}$
	\item $\langle\alpha,\alpha\rangle\geq 0$当且仅当$\alpha=\theta$时取等
	\item $\langle\alpha, \beta_1 k_1 +\beta_2 k_2\rangle = \langle\alpha,\beta_1 \rangle k_1+\langle \alpha, \beta_2 \rangle k_2$
\end{enumerate}
时
$\langle,\rangle$为内积,$\langle V,+,\cdot,\langle,\rangle \rangle$为内积空间。
注意
$\langle\beta_1 k_1 +\beta_2 k_2, \alpha\rangle = \langle\beta_1 \alpha\rangle \overline{k_1}+\langle\beta_2 \alpha\rangle \overline{k_2}$。
