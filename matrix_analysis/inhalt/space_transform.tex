\section{空间与变换}
\subsection*{线性空间}
对$V,+,\cdot$,在
\begin{enumerate}
	\item $(\alpha+\beta)+\gamma = \alpha + (\beta+\gamma)$
	\item $\alpha+\beta=\beta+\gamma$
	\item $\exists\theta\forall\alpha,\alpha+\theta=\alpha $
	\item $\forall\alpha\exists\gamma,\alpha+\gamma=\theta $
	\item $(\alpha\cdot k_1 )\cdot k_2 = \alpha\cdot(k_1k_2)$
	\item $\alpha\cdot(k_1+k_2)=\alpha\cdot k_1+\alpha\cdot k_2$
	\item $(\alpha_1+\alpha_2)\cdot k =\alpha_1\cdot k + \alpha_2\cdot k$
	\item $\alpha \cdot 1 = \alpha$
\end{enumerate}
中满足前四条的$\langle V,+ \rangle$为交换群,满足全部的$\langle V,+,\cdot \rangle$为线性空间。
\subsection*{维数、基底与坐标}
$\forall \beta\in V, \exists ! k_1\dots k_n\in F ,\beta = \textstyle\sum\limits_{i=1}^{n}\alpha_i k_i$
表示$n$维空间$V$下$\beta$在基底$\left(\begin{smallmatrix}
			\alpha_1\dots\alpha_n\end{smallmatrix}\right) $下的坐标为
$\left(\begin{smallmatrix}
			k_1\dots k_n\end{smallmatrix}\right)^T$。
如$V=\mathbb{R}^{2\times 3},F=\mathbb{R}$的一组基底是将$2\times 3$零矩阵逐个元素代入1得到的六个矩阵,
按照$\beta = \left(\begin{smallmatrix}
			\alpha_1\dots\alpha_n\end{smallmatrix}\right)
	\left(\begin{smallmatrix}
			k_1\dots k_n\end{smallmatrix}\right)^T$计算时可先将矩阵拉直为列向量。
\subsection*{换基底}
$\left(\begin{smallmatrix}
			\alpha_1\dots\alpha_n\end{smallmatrix}\right)P=\left(\begin{smallmatrix}
			\beta_1\dots\beta_n\end{smallmatrix}\right)$中$P$为基$(\alpha_{i})$到$(\beta_i)$的过渡矩阵。
于是$\beta=(\alpha_i)X=(\beta_i)Y=(\alpha_i)PX$导出$X=PY$。
\subsection*{内积空间}
对线性空间$\langle V,+,\cdot \rangle$和运算$\langle,\rangle$满足
\begin{enumerate}
	\item $\langle\beta,\alpha\rangle = \overline{\langle\alpha,\beta\rangle}$
	\item $\langle\alpha,\alpha\rangle\geq 0$当且仅当$\alpha=\theta$时取等
	\item $\langle\alpha, \beta_1 k_1 +\beta_2 k_2\rangle = \langle\alpha,\beta_1 \rangle k_1+\langle \alpha, \beta_2 \rangle k_2$
\end{enumerate}
时
$\langle,\rangle$为内积,$\langle V,+,\cdot,\langle,\rangle \rangle$为内积空间。
注意
$\langle\beta_1 k_1 +\beta_2 k_2, \alpha\rangle = \langle\beta_1, \alpha\rangle \overline{k_1}+\langle\beta_2, \alpha\rangle \overline{k_2}$。
\subsection*{内积示例}
复列向量内积$\langle x,y\rangle=y^Hx$,方阵的Frobenius内积$\langle A,B\rangle=tr(A^HB)=\sum_i\sum_j\overline{a_{ij}}b_{ij}$。
\subsection*{Gram矩阵(度量矩阵)}
$G=(\langle \alpha_i,\alpha_j\rangle)$是基$(\alpha_i)$的度量矩阵。满足Hermite对称$G^H=G$,Hermite半正定$x^HGx\geq 0$。
使用$P$将基$(\alpha_i)$变换为$(\alpha_i')$后,后者的度量矩阵$G'=P^HGP$。计算$\langle \alpha_i',\alpha_j' \rangle$即得。

\subsection*{子空间}
若$W$是$F$上线性空间$V$的非空子集且
\begin{enumerate}
	\item $\forall \alpha+\beta\in W,\alpha+\beta\in W$
	\item $\forall \alpha \in W, k\in F, \alpha k\in W$
\end{enumerate}
则称$W$是$V$中的一个线性子空间(线性可通过定义验证),记作$W\leq V$
最小的子空间是${\theta}$。
\subsection*{生成子空间}
任取$V$中非空子集$S$,其所有线性组合组成生成子空间$W=span S$,
这是$V$中包含$S$中全部元素的最小子空间。$V$是它的基的生成子空间。
\subsection*{子空间的和}
$W_1+W_2=\{ \alpha_1+\alpha_2 | \alpha_1\in W_1, \alpha_2\in W_2 \}\subseteq V$,
子空间的并是和的子集但不是线性空间因此不是子空间。
\subsection*{维数公式}
$dim(W_1)+dim(W_2)=dim(W_1+W_2)+dim(W_1\cap W_2)$,类似容斥原理。
\subsection*{直和分解}
若$W_1+W_2=V,W_1\cap W_2=\{\theta\}$则$V=W_1\oplus W_2$(直和)。
等价于$\theta=\alpha_1\in W_1+\alpha_2\in W_2$当且仅当$\alpha_1=\alpha_2=\theta$,
证明思路是假设有两种分解后反证。$W_1$和$W_2$的基的并是$V$的一组基,因为$W_1$的基和$W_2$的基线性无关。
若给定$V$的一个子空间,可通过求基的方式构造出与其直和得到$V$的另一个子空间。

\subsection*{线性映射}
线性映射$\varphi:V_1\rightarrow V_2$满足
\begin{enumerate}
	\item $\varphi(\alpha+\beta)=\varphi(\alpha)+\varphi(\beta)$
	\item $\varphi(\alpha\cdot k)=\varphi(\alpha)\cdot k$
\end{enumerate}
或写作$\varphi((\alpha_i)^T(k_i))=\varphi(\alpha_1\dots\alpha_n)(k_i)$。\\
$V_1=V_2=V$时称线性自映射(线性算子)。
\subsection*{线性映射的矩阵表示}
$\varphi:V_1\rightarrow V_2$,对于$V_1$的基$\alpha$和$V_2$的基$\beta$有
$\varphi(\alpha_1\dots\alpha_n)=(T(\alpha_i))=(\alpha_1\dots\alpha_n)A=(\beta_1\dots\beta_n)$
\subsection*{矩阵转换}
令$\varphi:V_1\rightarrow V_2$,$V_1$下基$\alpha$到$\alpha'$的过渡矩阵$P$,
$V_2$下基$\beta$到$\beta'$的过渡矩阵$Q$,
$\varphi$由$\alpha$变换为$\beta$时矩阵表示为$A$,由$\alpha'$变换为$\beta'$时矩阵表示为$B$。
计算可知$AP=QB,B=Q^{-1}AP$。
对于线性算子,取$\alpha=\beta,Q=P$即得到相似。
\subsection*{核}
$kernel\varphi=\{\alpha\in V_1|\varphi(\alpha=\theta_{V_2})\}\leq V_1$
要证子空间只需验证线性。$kernel\varphi=\{\theta_{V_1}\}$对应单射(不会有两个元素被映射到同一个元素,反证得到)。
\subsection*{像}
$image\varphi=\{\varphi(\alpha)|\alpha\in V_1)\}\leq V_2$
要证子空间只需验证线性。$image\varphi=V_2$对应满射。
\subsection*{正交补}
$W\leq V$,$W$的正交补$W^{\perp}=\{\beta\in V|\forall\alpha\in W,\langle\alpha,\beta\rangle=0 \}\leq V$。
满足$W+W^{\perp}=V$,因为将$W$的基扩充为$V$的基时新增基的生成子空间可证明与$W^{\perp}$互相包含。
$W\cap W^{\perp}=\{0\}$,故$W\oplus W^{\perp}=V$,由维数公式可知它们维数之和为$dimV$。
\subsection*{保内积算子和酉阵}
$\varphi:V\rightarrow V$满足$\langle\varphi(\alpha), \varphi(\beta)\rangle=\langle\alpha,\beta\rangle$则称为保内积算子。
取标准正交基代入定义计算,可知其变换矩阵$U^TU=I$为酉矩阵。
计算两组标准正交基的Gram矩阵后可得到它们的过渡矩阵是酉矩阵。
酉矩阵的各列向量组成一组标准正交基。
