\section{杂项}
\subsection*{数列}
构造合适的线性变换求矩阵的高次幂得到通项。
如$a_{n+2}+2a_n=3a_{n+1}+2^n$转换为
\[
	\begin{psmallmatrix}
		a_{n+2}\\a_{n+1}\\2^{n+1}
	\end{psmallmatrix}=
	\begin{psmallmatrix}
		3&-2&1\\1&0&0\\0&0&2
	\end{psmallmatrix}	\begin{psmallmatrix}
		a_{n+1}\\a_n\\2^n
	\end{psmallmatrix}
\]
\subsection*{一个$e^{At}$的例子}
$A$的特征多项式$p(x)=(x+2)(x-2)(x-1)^2$,
$f(x)=e^{xt}=q(x)p(x)+c_0+c_1x+c_2x^2+c_3x^3$,
代入$f(-2),f(2),f(1)$和$f'(1)=(te^{xt})|_{x=1}=te^t=c_1+c_2+c_3$求解。
\subsection*{关于$A^HA$和$AA^H$}
\begin{enumerate}
	\item $rankA^HA=rankAA^H=rankA$,一方面$Ax=0$显然$A^HAx=0$,另一方面$A^HAx=0$时$x^HA^HAx=0,\norm{Ax}^2=0$,因此$Ax=0$;而$rankAA^H=rankA^H=rankA$。
	\item $A^HA$和$AA^H$特征值全为非负实数,它们都是正规矩阵,在$x^HA^HAx=\norm{Ax}^2>0$中对$A^HA$用Schur可证。
	\item $A^HA$与$AA^H$的特征值相同,零时显然,非零时$A^HAx=x\lambda\Rightarrow A(A^HAx)=(Ax)\lambda$,$A^HA$的特征值全为$AA^H$的特征值,反之类似。
\end{enumerate}
