\section{范数理论}
\subsection*{线性空间V上的范数}
设$V$为$F$上的线性空间,若映射$\norm{\cdot}:V\rightarrow \mathbb{R}$
满足\begin{enumerate}
	\item $\norm{v}\geq 0,\norm{v}=0\iff v=\theta$
	\item $\norm{v\lambda}=\norm{v}|\lambda|$
	\item $\norm{v_1+v_2}\leq\norm{v_1}+\norm{v_2}$
\end{enumerate}
则称$\norm{\cdot}$为$V$上的一个$F$范数,$\{V,\norm{\cdot}\}$为赋范线性空间。
$d(v_1,v_2)=\norm{v_1-v_2}$称作$v_1,v_2$的距离,$\{V,d\}$为度量线性空间。
$d(v_1,v_3)=\norm{v_1-v_2+v_2-v_3}\leq d(v_1,v_2)+d(v_2,v_3)$。
\subsection*{内积诱导范数}
$\norm{v}=\sqrt{\langle v,v \rangle}$。对于
$\forall \alpha,\beta\in V$有$\iproduct{\alpha+\beta,\alpha+\beta}=\iproduct{\alpha,\alpha}+\iproduct{\alpha,\beta}+\iproduct{\beta,\alpha}+\iproduct{\beta,\beta}$,
$\iproduct{\alpha-\beta,\alpha-\beta}=\iproduct{\alpha,\alpha}-\iproduct{\alpha,\beta}-\iproduct{\beta,\alpha}+\iproduct{\beta,\beta}$,
两式相加得到$\norm{\alpha+\beta}^2+\norm{\alpha-\beta}^2=2\norm{\alpha}^2+2\norm{\beta}^2$。
\subsection*{勾股定理和Cauchy不等式}
对于内积诱导范数,展开计算可证明
\begin{enumerate}
	\item $\alpha\perp\beta\iff \norm{\alpha+\beta}=\norm{\alpha}+\norm{\beta}$。
	\item $\iproduct{\alpha,\beta}\iproduct{\beta,\alpha}=|\iproduct{\alpha,\beta}|\leq\norm{\alpha}^2\norm{\beta}^2$
\end{enumerate}
Cauchy不等式在$\alpha\neq\theta$时显然成立;其他情况构造$\alpha\perp\gamma=\beta-\alpha k$可证明,且当且仅当$\alpha$与$\beta$线性相关时等号成立。
$\alpha$和$\beta$的夹角满足$\iproduct{\alpha,\beta}=\norm{\alpha}\norm{\beta}\cos\lambda$。
\subsection*{列向量上的p-范数}
对于列向量$v$,$\norm{v}_p=(\sum|x_i|^p)^{\frac1p}$:
\begin{enumerate}
	\item $p=1:\sum|x_i|$
	\item $p=2:\sqrt{v^Hv}=\sqrt{\sum x_i^2}$
	\item $p=\infty:\lim_{p\to\infty}\norm{v}_p=\max\{|x_i|\}$
\end{enumerate}
注意2-范数保酉变换。
\subsection*{线性映射构造范数}
线性单射$\varphi:V_1\to V_2$,$\norm{\cdot}_{V_2}$为$V_2$上的范数,那么
$V_1$上定义$\norm{v}_\varphi=\norm{\varphi{v}}_{V_2}$为$V_1$上的范数。
例如,对矩阵使用拉直变换为列向量后可使用列向量的2-范数定义矩阵的Frobenius范数。
\subsection*{范数等价}
有限维空间中序列是否为Cauchy列与范数选取无关。
$\exists c_1,c_2>0,\forall v\in V \norm{v}_\alpha\leq c_1 \norm{v}_\beta,\norm{v}_\beta\leq c_2 \norm{v}_\alpha$。
有限维空间中Cauchy列一定收敛,因为可以归结为列向量空间上的$\infty$-范数,也就可以对每个矩阵元素取极限得到整个矩阵的极限。
\subsection*{线性映射的范数}
赋范线性空间$(V_2,\norm{\cdot}_\beta),(V_1,\norm{\cdot}_\alpha)$有线性映射$\varphi:V_2\to V_1$,
定义$\norm{\varphi(v)}_{\alpha,\beta}=\sup_{v\neq\theta}\frac{\norm{\varphi(v)_\alpha}}{\norm{v}_\beta}=\max_{\norm{v}_\beta=1}\norm{\varphi(v)}_\alpha$,
相当于最大缩放比。$\norm{\varphi(v)}_{\alpha,\beta}$是$(\varphi:V_2\to V_1)$上的范数。
称$\norm{\varphi(v)}_\alpha\leq \norm{\varphi}_{\alpha,\beta}\norm{v}_\beta$为相容性条件。
次乘性:若$\psi:(V_3,\norm{\cdot}_\gamma)\to (V_2,\norm{\cdot}_\beta),\varphi:(V_2,\norm{\cdot}_\beta)\to(V_1,\norm{\cdot}_\alpha)$,
则$\norm{\psi\circ\varphi}_{\alpha,\beta}\leq\norm{\varphi}_{\alpha,\beta}\cdot\norm{\psi}_{\beta,\alpha}$
\subsection*{常用矩阵范数}
\begin{enumerate}
	\item 列向量1-范数的诱导范数(列和范数)$\norm{A^{m\times n}}_{1,1}=\max\limits_{\norm{x}_1=1}\norm{Ax}_1=\max\limits_{1\leq i\leq n}\sum\limits_{i=1}^m|a_{ij}|$
	\item 列向量2-范数的诱导范数(谱范数)$\norm{A^{m\times n}}_{2,2}=\max\sigma$,SVD可证。
	\item 列向量$\infty$-范数的诱导范数(行和范数)$\norm{A^{m\times n}}_{\infty,\infty}=\max\limits_{1\leq i\leq m}\sum\limits_{j=1}^n|a_{ij}|$
	\item $m_1$范数,矩阵拉直为列向量后取1-范数 $\norm{A^{m\times n}}_{m_1}=\sum\sum|a_{ij}|$
	\item Frobenius范数,矩阵拉直为列向量后取2-范数\\ $\norm{A^{m\times n}}_F=\sqrt{\iproduct{A,A}_F}=\sqrt{tr(A^HA)}=\sqrt{\sum\sigma_i^2}$
	\item $m_{\infty}$范数,矩阵拉直为列向量后取$\infty$-范数再乘矩阵尺寸\\ $\norm{A^{m\times n}}_{m_\infty}=\max\{m,n\}\max\limits_{i,j}|a_ij|$
\end{enumerate}
\subsection*{谱半径估计}
谱半径$\rho(A)$是$A$全部特征值绝对值中的最大值。
$\rho(A)\leq\norm{A}$,计算$Ax=x\lambda$范数可知。
$\rho(A)\geq\norm{A}+\epsilon$,证明思路为对$A$用Jordan,令$Q=diag(1,\epsilon,\epsilon^2\dots)$,
$A$的范数$\norm{Q^{-1}P^{-1}APQ}_{\infty,\infty}$满足条件。
因此$\rho(A)\leq 1\iff \exists \norm{\cdot}_*, \norm{A}_*\leq_1$。
\subsection*{矩阵幂收敛条件}
$\norm{A}\leq 1\Rightarrow k\to\infty, 0\leq \norm{A^k-0} = \norm{A^k}\leq \norm{A}^k = 0\Rightarrow \lim_{k\to\infty}A^k=0$
