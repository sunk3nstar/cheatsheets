\section{矩阵方程}
\subsection*{方程的有解性}
\begin{enumerate}
	\item $rank(A,\beta)=rank(A)$
	\item $\beta$在A像空间内
\end{enumerate}
若要对$\forall\beta$有解,$x\to Ax$是线性满射,行满秩$rankA=m$。
\subsection*{至多有一个解}
$x\to Ax$是线性满射,列满秩$rankA=n$,$A$各列线性无关。
\subsection*{左逆、右逆}
对于$A\in \mathbb{C}^{m\times n}$:
\begin{enumerate}
	\item $\exists G\in\mathbb{C}^{n\times m},AG=I_{m}\iff rankA =m$
	\item $\exists H\in\mathbb{C}^{n\times m},HA=I_{n}\iff rankA =n$
	\item 构造$HAG$可由1、2导出$\exists G,H,AG=I_{m},GA=I_n\iff rankA =m=n$
\end{enumerate}
1、2显然等价。要证明1,注意到$m\geq rankA\geq rank(AG)=rankI_m=m$得充分性,
若$rankA=m$,$Ax=I$有解,解即为$G$,必要性得证。
\subsection*{广义逆(m-p逆,伪逆)}
$G=A^\dagger$满足
\begin{tasks}(2)
	\task $AGA=A$
	\task $GAG=G$
	\task $(AG)^H=AG$
	\task $(GA)^H=GA$
\end{tasks}
满足第$i,j,\dots$条记作$A^{\{i,j,\dots\}}$。
使用SVD可得到$A=Q\begin{psmallmatrix}
		\Sigma_r&0\\0&0
	\end{psmallmatrix} P^{-1}$,$A^\dagger=P\begin{psmallmatrix}
		\Sigma_r^{-1}&0\\0&0
	\end{psmallmatrix} $
\subsection*{广义逆的满秩分解算法}
设满秩分解$A^{m\times n}=F^{m\times r}G^{r\times n}$,
则$A^{\dagger}=G^H(GG^H)^{-1}(F^HF)^{-1}F^{H}$,正确性通过代入定义验证。
\subsection*{最小二乘解}
对$Ax\approx \beta$,记$A^\dagger=G,x_0=G\beta,\gamma=\beta-Ax_0=(I-AG)\beta$,
$\beta=Ax_0+\gamma$由$Image(AG)$与$Image(I-AG)$组合得到,
全部最小二乘解$z=x_0+(I-GA)y_0,y_0\in\mathbb{C}^n$。
计算可知$Ax_0\perp\gamma$,于是$\norm{\gamma}_F=\min\norm{\beta-Ax}_F$。
$x_0G\beta$是极小范数的最小二乘解。
\subsection*{最小范数解}
对$Ax= \beta$,记$A^\dagger=G,x_0=G\beta$,
取解$\xi$,$GA(x_0-\xi)=0$,$\xi=\xi-x_0+x_0$由$Image(I-GA)$与$Image(GA)$组合得到,
通解$z=x_0+(I-GA)y_0,y_0\in\mathbb{C}^n$。
计算可知$(I-GA)\delta \perp GA\delta$,于是$\norm{x_0}_F=\min\limits_{Ax=\beta}\norm{x}_F$

\section{例:$AX-XB=C$的研究}
\subsection*{尺寸}
方程尺寸由$X\in\mathbb{C}^{m\times n}$唯一确定。
\subsection*{解的研究}
$\varphi: X\to AX-XB$是线性算子,拉直得到其矩阵表示尺寸为$mn\times mn$。
若$\forall C,\exists X,AX-XB=C$,那么$\varphi$是满射,作为方阵行列同时满秩,$\varphi$亦是单射,$ker\varphi=\{0\}$,$AX-XB=0$只有零解。
设$A,B$有公共特征值$\mu\in\mathbb{C}$,取特征向量$A\alpha=\alpha\mu,B^T\beta=\beta\mu$,令$x_0=\alpha\beta^T$可证明是$AX-XB=0$的一个非零解。
若$A,B$无公共特征值,取$\varphi_1:X\to AX,\varphi_2: X\to BX,\varphi=\varphi_1-\varphi_2$。
求$\varphi_1$和$\varphi_2$的矩阵表示,在$m=2,n=3$时$\varphi$的矩阵表示$L=A\otimes I_3- I_2\otimes B$,
取$A,B$为对应阶Jordan块计算$\det (L)$可知非零,$AX-XB=0$只有零解。对原方程配Jordan标准型可发现一般情况仍然成立。
\subsection*{应用}
$AX-XB=C$有解$\Rightarrow$\[ \exists X,\begin{psmallmatrix}
		I&X\\O&I
	\end{psmallmatrix}\begin{psmallmatrix}
		A&C\\O&B
	\end{psmallmatrix}\begin{psmallmatrix}
		I&-X\\O&I
	\end{psmallmatrix}=\begin{psmallmatrix}
		A&C-(AX-XB)=O\\O&B
	\end{psmallmatrix}\]即$\begin{psmallmatrix}
		A&C\\O&B
	\end{psmallmatrix}\sim
	\begin{psmallmatrix}
		A&O\\O&B
	\end{psmallmatrix}$。
